\Nadpis{Analýza požadavků}

Cílem projektu je vytvořit nový programovací jazyk, jehož překladové jádro se bude skládat z algoritmů založených na umělé neuronové síti, které bude generovat výsledný strojový kód, který bude optymálnější než strojový kód generovaný konvenčními algoritmy, používanými s současných překladačích. Programovací jazyk se je kódově nazýván CC. Jedná se o snahu evokovat vztah k jazyku C.

Nově vytvořený programovací jazyk je kompatibilní s jazykem C, který je možné využít jako doplňující jazyk. To spočívá především ve využití již existujících knihoven pro jazyk C. Díky tomu velmi vzroste použitelnost nově vzniklého programovacího jazyka CC, který s minimální námahou bude možné v plné míře využít se stávající sadou knihovne. 

Proces překladu se bude skládat z několika fází ekvivalentních k překladu u C kompatibilních překladačů:

\vskip 4mm
\bod {Lexikální analýza}
\bod {Preprocesing}
\bod {Sémantická analýza}
\bod {Syntaktická analýza}
\bod {Asembler}
\bod {Překlad do objektového kodu}
\bod {Linkování}
\vskip 4mm

\Sekce {Preprocesor}
Veškeré příkazy pro {\bf preprocesor} jsou psány velkými písmeny. Ve fázi preprocesingu jsou tato klíčová slova ze zdrojového kódu odfiltrována a interpretována. Seznam klíčových slov pro preprocesor jsou z větší části totožná s jazykem C:

\vskip 4mm
\bod{(DEFINE jméno\_makra hodnota) - vytvoření makra}
\bod{(IF podmínka) - podmínka pro pomíněný překlad}
\bod{(ELSIF podmínka) - vícenásobné větvění pro podmíněný překlad}
\bod{(ELSE) - větvení pro podmíněný překlad}
\bod{(ENDIF) - ukoční vypnuté zpracování zdrojových kódů
\bod{(INCLUDE jméno) - pokud je název souboru zapsán bez úvozovek hledá se v systémových adresářích s knihovnami, pokud je soubor zapsán s úvozovkami hledá se v lokálním adresáři}
\bod{(DEFINED jméno\_makra) - určí zda dané makro je definované příkazem (DEFINE) nebo neexistuje, vrací hodnotu 1 nebo 0}
\bod{(* komentář *)}

Kromě toho preprocesor vymaže všechny bílé znaky (přebytečné mezery, odřádkování, komentáře), vložení zdrojového kódu ze všech zevedených hlavičkových souborů a připraví zdrojový kód na lexikální analýzu.

Při preprocesingu se zavádění a interpretace maker, které jsou v nich obsažené vykonává rekurzivně.

\vskip 4mm


